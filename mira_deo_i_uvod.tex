% !TEX encoding = UTF-8 Unicode
\documentclass[a4paper]{article}

\usepackage{color}
\usepackage{url}
\usepackage[T2A]{fontenc} % enable Cyrillic fonts
\usepackage[utf8]{inputenc} % make weird characters work
\usepackage{graphicx}

\usepackage[english,serbian]{babel}
%\usepackage[english,serbianc]{babel} %ukljuciti babel sa ovim opcijama, umesto gornjim, ukoliko se koristi cirilica

\usepackage[unicode]{hyperref}
\hypersetup{colorlinks,citecolor=green,filecolor=green,linkcolor=blue,urlcolor=blue}

\usepackage{listings}

%\newtheorem{primer}{Пример}[section] %ćirilični primer
\newtheorem{primer}{Primer}[section]

\definecolor{mygreen}{rgb}{0,0.6,0}
\definecolor{mygray}{rgb}{0.5,0.5,0.5}
\definecolor{mymauve}{rgb}{0.58,0,0.82}

\lstset{ 
  backgroundcolor=\color{white},   % choose the background color; you must add \usepackage{color} or \usepackage{xcolor}; should come as last argument
  basicstyle=\scriptsize\ttfamily,        % the size of the fonts that are used for the code
  breakatwhitespace=false,         % sets if automatic breaks should only happen at whitespace
  breaklines=true,                 % sets automatic line breaking
  captionpos=b,                    % sets the caption-position to bottom
  commentstyle=\color{mygreen},    % comment style
  deletekeywords={...},            % if you want to delete keywords from the given language
  escapeinside={\%*}{*)},          % if you want to add LaTeX within your code
  extendedchars=true,              % lets you use non-ASCII characters; for 8-bits encodings only, does not work with UTF-8
  firstnumber=1000,                % start line enumeration with line 1000
  frame=single,	                   % adds a frame around the code
  keepspaces=true,                 % keeps spaces in text, useful for keeping indentation of code (possibly needs columns=flexible)
  keywordstyle=\color{blue},       % keyword style
  language=Python,                 % the language of the code
  morekeywords={*,...},            % if you want to add more keywords to the set
  numbers=left,                    % where to put the line-numbers; possible values are (none, left, right)
  numbersep=5pt,                   % how far the line-numbers are from the code
  numberstyle=\tiny\color{mygray}, % the style that is used for the line-numbers
  rulecolor=\color{black},         % if not set, the frame-color may be changed on line-breaks within not-black text (e.g. comments (green here))
  showspaces=false,                % show spaces everywhere adding particular underscores; it overrides 'showstringspaces'
  showstringspaces=false,          % underline spaces within strings only
  showtabs=false,                  % show tabs within strings adding particular underscores
  stepnumber=2,                    % the step between two line-numbers. If it's 1, each line will be numbered
  stringstyle=\color{mymauve},     % string literal style
  tabsize=2,	                   % sets default tabsize to 2 spaces
  title=\lstname                   % show the filename of files included with \lstinputlisting; also try caption instead of title
}

\begin{document}

\title{Naslov seminarskog rada\\ \small{Seminarski rad u okviru kursa\\Metodologija stručnog i naučnog rada\\ Matematički fakultet}}

\author{Prvi autor, drugi autor, treći autor, četvrti autor\\ kontakt email prvog, drugog, trećeg, četvrtog autora}

%\date{9.~april 2015.}

\maketitle

\tableofcontents

\newpage

\section{Uvod}
\label{sec:uvod}


Koncept veštačke inteligencije iako deluje moderno zapravo postoji veoma dugo. Sama ideja se javlja još u periodu antičkog doba. Kroz istoriju ljudima je uvek bila zanimljiva ideja pravljenja mašine koja bi bila slična čoveku. U modernoj istoriji to se može videti kroz kulturno stvaralaštvo gde se veoma često provlači ta ideja, najčešće kroz filmove. Jedan od interesantnijh primera je film Čarobnjak iz Oza (1939).

Prvi ozbiljniji radovi na ovu temu potiču iz sredine XX veka. Najpoznatiji su Tjuringov članak `Computing Machinery and Intelligence` [?], Šenonov rad `Programming a Computer for Playing Chess`[?], rad Džona Mekartija `Why Artificial Intelligence Needs Philosophy`[?] za koji je dobio Tjuringovu nagradu 1971. godine. 

Poslednjih godina se nauka u oblasti računarstva i matematike dovoljno razvila da je tehnički moguće napraviti mašine koje samostalno donose odluke. Medjutim, glavnu prepreku za dalji razvoj i širu upotrebu istih predstavlja etika. Uvodjenje etički ispravnog ponašanja kod ovakvih mašina predstavlja veliki problem jer je za početak potrebno definisati jedinstveni etički okvir na osnovu kog bi se implementiralo ponašanje tih mašina. Definisanje takvog okvira je vema težak zadatak i praktično je nemoguće da se ljudi dogovore šta je etički ispravno ponašanje, jer u različitim državama vladaju različiti etički i pravni zakoni.

U ovom radu detaljno su obrajdene četiri teme, veštačka inteligencija i društvene mreže, autonomna vozila, veštačka inteligencija u medicini i veštačka inteligencija u vojne svrhe. Kao referenca koja je korišćena za veliki deo rada izdvaja se knjiga `ta i ta`[??].



\section{Veštačka intelijencija u medicini}
\label{sec:upotreba_veštačke_intelijencije_u_medicini}

Veštačka inteligencija pronašla je svoju primenu i u medicini. Primenjuje se u raznim njenim specijalnostima poput dermatologije, radiologije, traumatologije, onkologije, gastroenterologije, dijabetesa, bioinženjeringa, oftamoogije. U nekim od ovih oblasti veštačka inteligencija se pokazala manje ili više uspešnom, ali za sve njih zajedničko je da je cilj primnene veštačke inteligencije bilo olakšanje i poboljšanje procesa  posmatranja, dijagnoze i na kraju samog lečenja bolesti kod obolelih pacijenata. Dok se kao vodeća specijalnost u kojoj se najuspešnije primenjuje veštačka inteligencija izdvaja se oftamologija.

Upotreba veštačke inteligencije u medicini donela je i veliki broj etičkih problema. Naime, pri početku njenog upotrebljavanja  kao novog trenda u medicinu nije delovalo da postoje bilo kakvi razlozi za diskusiju po pitanju etike u upotrebi veštačke inteligencije u medicini, medjutim kako je vreme prolazilo postalo je očigledno da postoje veliki etički probemi u tom polju.

Neki od najistaknutijih etičkih problema u primeni VI u medicini opisani su detaljno u narednim poglavljima. Pored problema istaknutih u nastavku postoje i mnogo drugi, ali izdvojeni su samo najinteresantniji.

\subsection{Potrebni podaci}
\label{subsec:poreklo_podataka}

Kako bi se veštačka inteligencija uspešno upotrebljavala u medicini potrebno je obezbediti veliku količinu podataka. Ti podaci su potrebni algoritmima veštačke inteligencije kako bi trenirali na njima i kako bi se vršila validacija rezultata rada tih algoritma, medjutim tu postoji problem. Naime, potrebni podaci potiču iz različitih resursa poput elektronskih zdravstvenih kartona pacijenata, kliničkih istraživanja i slično. Sa takvim vrstama podataka postoji više različitih etičkih problema, poput vlasništva i javne upotrebe.

Vlasništvo medicinskih podataka predstavlja etički problem jer onaj koji ga poseduje ima moć kontrole, pristupa i obrade tih podataka, a takodje može ostvariti i profit od prava na prodaju tih podataka. Medjutim, nije u potpusnoti definisano ko je vlasnik tih podataka. Pored pacijenata čiji su podaci u pitanju pravo na vlasništvo tih podataka mogu tražiti i zdrastvene ustanove u kojima se pacijent leči, lekari, zdravstveni osiguravači pacijenta, korporacije ili pojedinci koji su odgovorni za generisanje, skladištenje i obradu podataka i mnogi drugi.

Ukoliko bi problem vlasništva nad podacima bio rešen, na scenu stupa novi problem, a to je njihova javna upotreba. Da li je potrebno imati dozvole vlasnika podataka da se ti podaci javno koriste, da li je etički prihvatljivo javno korititi te podatke bez pristanka vlasnika?

Korišćenjem tih podataka bez odgovarajućih dozvola mogu nastati razni problemi. Neki od njih su emocionalni stres zbog izlaganja osetljivih zdravstvenih podataka, diskriminacija, lišavanje zdravstvenog osiguranja ili zaposlenja, netraženje zdravstvenih usluga ili uskraćivanje informacija radi zaštite privatnosti i slično.

Ni jedan od ova dva problema nije precizno i uniformno svugde u svetu rešen. Ideja koja se nameće je da bi oni mogli biti rešeni uvodjenjem odgovarajućih zakonskih regulativa koje su u skladu sa etički prihvatljivim ponašanem, i danas se zaista radi na tome u svetu. Medjutim, i dalje je sve vezano za tu oblast u povoju i potrebno je dosta rada u tom polju kako bi se ova dva problema, koja se zapravo prepliću uspešno rešila, a samim tim i ubrzao razvoj veštačke inteligencije. \cite{ai_in_medicine}

\subsection{Veštačka inteligencija i lekari}
\label{subsec:veštačka_inteligencija_i_lekari}

Prilikom pojave svake nove tehnologije dolazi do stvaranja odredjenog stepena zavisnoti od te tehnologije. Tako se pojavila i delimična zavisnost lekara od veštačke inteligencije, a ta zavisnost može dovesti do raznih problema. Može se desiti da se lekari previše oslanjaju na rešenja koja im predlaže veštačka inteligencija i time postanu nedovoljno oprezni, ili na duže staze može dovesti do gubitka veštine. Takodje, s obzirom na to da su danas veštine lekare potpuno ili delimično zamenjene tehnologijom dolazi do gubitka samopouzdanja i kompetencije lekara.

Pored zavisnoti lekara od veštačke inteligencije pojavilo se i suparništvo izmedju njih i veštačke inteligencije. Naime, kao što je ranije i navedeno, veštačka inteligencija se uvela u medicinu sa ciljem da olakša i unapredi rad lekara. Medjutim, i u medicini se kao i u raznim drugim obastima stvara strah od preuzimanja posla od ljudi od strane veštačke inteligencije. Pored problema zavisnoti i suparništva, ističu se i problemi poverenja pacijenata, saosećanja i empatije prema pacijentima.

Poverenje koje pacijenti danas imaju u lekare zasniva se na hiljadugodišnjoj praksi. Dok se sa druge strane smatra da za sticanje poverenja u veštačku inteligenciju od strane pacijenata neće biti potrebno toiko vremena, ali ono danas i dalje ne postoji. Smatra se da je i dalje potrebno dosta rada i truda kako bi se pacijenti uverili u bezbednost primene veštačke inteligencije. Glavni razlog zbog kojih pacijenti i dalje nemaju poverenje u veštačku inteligenciju su pristrasna, netačna, neefikasna, neobjašnjiva i netransparenta rešenja koja u nekim situacijama pruža veštačka inteligencija.

Lekari za razliku od mašina i veštačke inteligencije imaju saosećanje i osećaj empatije prema pacijentima. Ova činjecice se takodje smatra jednim od glavnih razloga zašto se veštačke inteligencije i daje ne prihvata medju pacijentima.

Potrebno je empatične veštine i znanja inkorporirati u programe veštačke inteligencije u medicini. Ideja je da se unošenjem hiljada scenarija pacijenta u algoritmime veštačke inteligencije oni nauče da imaju empatične reakicije. Tj. ideja je da se razvije "veštačka naklonost" mašina, da mašine nauče da osete i izraze bol i osećanja, tj. da se poboljša "ličnost" mašina. Ovim akcijama mašine bi razvile sposobnost saosećanja sa pacijentima, ali osećaj odgovornosti za štetu nanentu njihovim radnjama.

\end{document}
