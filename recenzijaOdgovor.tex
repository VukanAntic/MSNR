% !TEX encoding = UTF-8 Unicode

\documentclass[a4paper]{report}

\usepackage[T2A]{fontenc} % enable Cyrillic fonts
\usepackage[utf8x,utf8]{inputenc} % make weird characters work
\usepackage[serbian]{babel}
%\usepackage[english,serbianc]{babel}
\usepackage{amssymb}

\usepackage{color}
\usepackage{url}
\usepackage[unicode]{hyperref}
\hypersetup{colorlinks,citecolor=green,filecolor=green,linkcolor=blue,urlcolor=blue}

\newcommand{\odgovor}[1]{\textcolor{blue}{#1}}

\begin{document}

\title{Etički problemi u veštačkoj inteligenciji\\ \small{Vukan Antić, Katarina Dimitrijević, Mirjana Jočović, Aleksandar Šarbajić}}

\maketitle

\tableofcontents


\chapter{Recenzent \odgovor{--- 5:} }


\section{O čemu rad govori?}
% Напишете један кратак пасус у којим ћете својим речима препричати суштину рада (и тиме показати да сте рад пажљиво прочитали и разумели). Обим од 200 до 400 карактера.
Objašnjava šta su sistemi preporučivanja i kako ih koriste društvene mreže. Navodi probleme povezane sa privatnošću medicinskih  podataka. Pokriva probleme implementacije etički ispravnih odluka u sistemima za autonomno upravljanje vozilima, medicini i kreiranju i korišćenju oružja. Glavno pitanje je prisustvo ljudskog faktora u navedenim situacijama i da li ga je moguće eliminisati.

\section{Krupne primedbe i sugestije}
% Напишете своја запажања и конструктивне идеје шта у раду недостаје и шта би требало да се промени-измени-дода-одузме да би рад био квалитетнији.
Poglavlje o društvenim mrežama ne spominje TikTok koji predstavlja trenutno najpopularniju drušvenu mrežu, posebno među decom i mladima, a poseduje najrazvijeniji algoritam za preporučivanje sadržaja. Smatram da bi bilo dobro ukratko prokomentarisati i ovu društvenu mrežu na sličan način kao preostale dve. 

\odgovor{Posle dužeg razmišljanja, pri izradi 2. pogljavlja rada, došli sam do zaključka da nije neophodno dodavati poglavlje koje se bavi Tiktokom (eng.~{\em Tiktok}). Naime, nismo uspeli da pronađemo nikakve primedbe koje se tiču etike njihovog algoritma (ne sumnjamo da postoje), a i pored toga, Jutjubov (eng.~{\em YouTube}) algoritam funkcioniše isto kao Tiktokov, pa smatramo da nema smisla dva puta pričati o tome.}

\section{Sitne primedbe}
% Напишете своја запажања на тему штампарских-стилских-језичких грешки
\begin{enumerate}
\item Naslov poglavlja broj 2: jedino ovde se veštačka inteligencija pojavljuje u množini, nije konzistentno sa ostatkom sadržaja i rada generalno. \\
\odgovor{Ispravljeno je na ''Veštačka inteligencija'', da se poklapa sa ostatkom teksta.}

\item Uvod: navođenje naslova radova bi trebalo da bude pod dvostrukim navodnicima \\
\odgovor{Ispravljeno je navođenje naziva radova svuda u tekstu - koriste se dvostruki navodnici.}

\item Poglavlje 2, prvi pasus: slovna greška u reči ''preporučivanja'' \\
\odgovor{Ispravljena je slovna greška.}

\item Poglavlje 2.1.1, prvi pasus: slovna greška u reči ''pretražuje'' \\
\odgovor{Ispravljena je slovna greška.}

\item Poglavlje 2.1.1, treći pasus: treba ''uspavanki'' umesto ''uspavanka'' \\
\odgovor{Ispravljena je slovna greška.}

\item Poglavlje 2.1.2: umesto stranog termina ''playlist-e'' bi trebalo koristiti adekvatan prevod (npr. liste pesama, liste numera) ili napisati samo ''plejliste'' \\
\odgovor{Ispravljeno je na liste pesama.}


\item Poglavlje 2.2, prvi pasus: strani termin ''persona'' može se zameniti sa npr. ''osoba'' \\
\odgovor{Ispravljeno je po predlogu.}

\item Poglavlje 2.2, prvi pasus: ispravno je ''dajući'' umesto ''davajući'' \\
\odgovor{Ispravljena je gramatička greška.}

\item Poglavlje 2.2, drugi pasus: ''zavisnost se razvija \textbf{kod}'' umesto ''zavisnost se razvija \textbf{u}''; deo rečenice ''... svakom korisniku može doći do ovih osećanja'' - nije baš gramatički ispravno, a nije ni u duhu jezika. \\
\odgovor{Ispravljena je da bude više u duhu jezika na ''Iako ova zavisnost nije podjednako zastupljena kod svih koji koriste društvene mreže, većina njih mogu da iskuse njene negativne posledice''.}

\item Poglavlje 2.2, drugi pasus: termin ''telefonija'' ne odgovara kontekstu rečenice i treba izbaciti zarez ispred ''ili'', npr. ''Lečenje ove bolesti podrazumeva smanjenje
korišćenja ovih platformi ili elektronskih uređaja uopšte.'' \\
\odgovor{Ispravljena po predlogu.}

\item Naziv Tabele 1: ''izražen'' umesto ''izraženo'' ili dodati zarez ispred ''izraženo''; ova tabela nije nigde referisana u tekstu. \\
\odgovor{Dodata je referenca na tabelu i ispravljena je po predlogu na izražen.}

\item Poglavlje 3, prvi pasus: ''... koji vozilo poseduje ...'' umesto ''... koji poseduje vozilo ...'' (drugačiji kontekst) \\
\odgovor{Izmenjeno je - stilski je ispravnije predloženo.}

\item Poglavlje 3, drugi pasus: slovna greška u reči ''Poslednjih'' \\
\odgovor{Ispravljena je slovna greska.}

\item Slika 1 bi prema nazivu trebalo da prikaže broj patenata po državama, ali se na samoj slici taj broj ne vidi. Trebalo bi dodati nekakvu skalu na slici ili promeniti njen naziv. \\
\odgovor{Slažemo se, međutim mislimo da sami brojevi nisu od preteranog značaja samom čitaocu, već da mu je sasvim dovoljno slikovito da vidi grafički odnos brojeva koji su vodeće države u ovoj oblasti dostigle i da je u redu izmeniti opis slike na sledeći način -  Odnos broja patenata koje su razvile države u oblasti autonomne vožnje.}

\item Poglavlje 3.1: ispravno je ''poznajući'' umesto ''poznavajući'' \\
\odgovor{Izmenjeno je - gramatički je ispravno predloženo.} 

\item Poglavlje 3.4: ispravno je ''kontroverznih'' umesto ''kontraverznih'' \\
\odgovor{Ispravljeno je.}

\item Poglavlje 4, prvi pasus: slovna greška ''cilj primene''

\odgovor{Ispravljena je slovna greška.} 

\item Poglavlje 4, prvi pasus: imam utisak da bi trebalo navesti neku referencu ili primer kako se veštačka inteligencija koristi u oftamologiji i odakle potiče podatak da je ona vodeća specijalnost 

\odgovor{Slažemo se, dodata je referneca u tekstu na osnovu koje iznosimo tvrdnju da se oftamologija ističe kao vodeća specijalnost u kojoj se primenjuje veštačka inteligencija. Takođe, dodata je još jedna rečenica koja čitaocu daje ideju zašto je to slučaj, međutim primer primene nismo dodali jer smatramo da to nije tema ovog rada. To su tehnički aspeti primene veštačke inteligencije u medicini dok je tema rada etika primene veštačke inteligencije u medicini i nisamo dovoljno upućeni u to polje da bismo mogli pisati o tome.} 

\item Poglavlje 4, treći pasus: slovna greška ''mnogi drugi''

\odgovor{Ispravljena je slovna greška.}

\item Poglavlje 4.1, treći pasus: slovna greška ''koristiti''

\odgovor{Ispravljena je slovna greška.}

\item Poglavlje 4.1, peti pasus: slovna greška ''ponašanjem''

\odgovor{Ispravljena je slovna greška.}

\item Poglavlje 4.2, četvrti pasus: slovna greška ''činjenica''

\odgovor{Ispravljena je slovna greška.}

\item Poglavlje 4.3, četvrti pasus: slovna greška ''i dalje''

\odgovor{Ispravljena je slovna greška.}

\item Poglavlje 5, drugi pasus: ispravno je ''ljudskih'' umesto ''ljutskih'' (D ne prelazi u T ispred S i Š; ista greška u poglavlju 5.2 u reči ''predstavlja'')

\odgovor{Ispravljena je slovna greška.}

\item Poglavlje 5.1 (i svuda gde se pojavljuje ovaj termin): mislim da reč ''pretprilagođenost'' ne postoji, a ne razumem dovoljno kontekst da predložim drugi termin.

\odgovor{Ispravljena je slovna greška, promenjeno na ˝preprilagođenost˝.}

\item Poglavlje 5.1.1: nedostaje zarez posle ''Naime''

\odgovor{Ispravljena se greška sa interpunkcijom.}

\item Poglavlje 5.1.2: kontradikcija - ''krupne detalje''

\odgovor{Ispravljena kontradikcija.}

\item Poglavlje 5.2.1, treći pasus: nigde drugde u radu se ne koristi prvo lice, zbog konzistentnosti predlažem da se ''možemo'' zameni sa ''može se''

\odgovor{Promenjena je reč zarad konzistentnosti.}

\item Zaključak, drugi pasus: slovna greška u reči ''veštačkoj''

\odgovor{Ispravljena je slovna greška.}

\item Zaključak, drugi pasus: ispraviti deo rečenice ''... neki od kojih su očigledniji od drugih.''

\odgovor{Rečenica je preformulisana.}

\item Pozicioniranje referenci u tekstu treba izmeniti, referenca treba da ide pre tačke i ne na kraju pasusa/poglavlja, već u odgovarajućoj rečenici na koju se odnosti \\
\odgovor{Izmenjeno je - reference treba navodoti na kraju rečenice, ali pre tačke. Međutim, nekada je ispravno navesti i na kraju pasusa, nije neophodno uvek da se odnosi na konkretnu rečenicu.}

\item Referenca [3] se pojavljuje samo u uvodu, a rečeno je da je korišćena za veći deo rada; trebalo bi referisati ovu knjigu na svakom mestu u radu u kom se koriste činjenice/tvrđenja iz nje; 

\odgovor{Dodate su odgovarajuće reference na [3] u tekstu.}

\item ispred ''i'' ne bi trebalo stavljati zarez, ovo se pojavljuje na par mesta

\odgovor{Na par mesta je izbačeno ˝i˝ iza kojeg je zarez, a na par mesta je ostavljeno jer je u ulozi intenzifikatora, što je gramatički ispravno.}

\end{enumerate}


\section{Provera sadržajnosti i forme seminarskog rada}
% Oдговорите на следећа питања --- уз сваки одговор дати и образложење

\begin{enumerate}
\item Da li rad dobro odgovara na zadatu temu?\\
Da, suština rada se odnosi na etičke probleme povezane sa razvojem veštačke inteligencije i pokriva više različitih aspekata problema.

\item Da li je nešto važno propušteno?\\
Ne, pokrivena su sva pitanja zadata opisom teme, kao i neka druga koja su takodje važna.

\item Da li ima suštinskih grešaka i propusta?\\
Ne, tema je dobro obrađena, rad je uglavnom napisan u skladu sa opštim smernicama.

\item Da li je naslov rada dobro izabran?\\
Da, potpuno odgovara temi i sadržaju rada.

\item Da li sažetak sadrži prave podatke o radu?\\
Da, konkretno i koncizno opisuje sadržaj rada.

\item Da li je rad lak-težak za čitanje?\\
Rad je relativno lak za čitanje, ima tok, ne sastoji se samo od nabrajanja činjenica, za svaku temu je napisana suština, bez preteranog ulaženja u detalje.

\item Da li je za razumevanje teksta potrebno predznanje i u kolikoj meri?\\
Potrebno je predznanje samo na osnovnom nivou, početak svakog poglavlja uvodi čitaoca u osnovne koncepte i terminologiju važnu za dalje čitanje.

\item Da li je u radu navedena odgovarajuća literatura?\\
Korišćena literatura je adekvatna; dosta su korišćene veb stranice i blogovi, ali su njihovi autori relevantni za datu temu

\item Da li su u radu reference korektno navedene?\\
Ne, potrebno je promeniti poziciju referenci u tekstu i na pojedinim mestima potkrepiti tekst referencama; ukoliko su podaci iz jednog rada korišćeni za više delova, potrebno je svuda ih referisati (npr. reference [1], [3], [4])\\
\odgovor{Izmenjeno je, ranije nisu pravilno navođene reference i nije ih bilo dovoljno.}

\item Da li je struktura rada adekvatna?\\
Da, podela po poglavljima je potpuno smislena, poglavlja su otprilike jednake veličine i podjednako značajna.

\item Da li rad sadrži sve elemente propisane uslovom seminarskog rada (slike, tabele, broj strana...)?\\
Da, sadrži i slike i tabelu, broj strana nije prekoračen, na prvoj strani se nalaze naslov, sažetak i sadržaj. 

\item Da li su slike i tabele funkcionalne i adekvatne?\\
Sliku 1 je potrebno dopuniti, Tabela 1 nije nigde referisana, a ostalo je u redu. \\
\odgovor{Izmenjen je opis slike kako bi odgovarao njenom sadržaju. Smatramo da nema potrebe dodavati brojeve jer nisu od preteranog značaja čitaocu, već odnos brojeva koji su vodeće države u ovoj oblasti dostigle, što je jasno na osnovu grafika. Takođe, dodata je referenca na tabelu.}
\end{enumerate}

\section{Ocenite sebe}
% Napišite koliko ste upućeni u oblast koju recenzirate: 
% a) ekspert u datoj oblasti
% b) veoma upućeni u oblast
% c) srednje upućeni
% d) malo upućeni 
% e) skoro neupućeni
% f) potpuno neupućeni
% Obrazložite svoju odluku
U oblast veštačke inteligencije sam veoma upućena, pre svega zbog fakulteta, u teme koje pokrivaju poglavlja 2 i 3 sam srednje do veoma upućena, a poglavlja 4 i 5 veoma malo, skoro neupućena. 


\chapter{Recenzent \odgovor{--- 5:} }


\section{O čemu rad govori?}
% Напишете један кратак пасус у којим ћете својим речима препричати суштину рада (и тиме показати да сте рад пажљиво прочитали и разумели). Обим од 200 до 400 карактера.

Govori o primeni i etičkim problemima veštačke inteligencije. U okviru društvenih mreža, tema su zavisnost i princip zasnovan na sistemu preporuke. U automobilskoj industriji je problem etika donošenja odluka kod autonomnog sistema upravljanja. U medicini su teme prava deljenja podataka i uticaj na lekare. U vojsci je dilema dodeljivanje odgovornosti i model učenja.


\section{Krupne primedbe i sugestije}
% Напишете своја запажања и конструктивне идеје шта у раду недостаје и шта би требало да се промени-измени-дода-одузме да би рад био квалитетнији.
Nedostaje navođenje referenci u teksu. Jednom je spomenuta referenca na knjigu, a kasnije u tekstu nigde više nije spomenuta kao referenca, zbog čega se pojavljuju tvrdnje za koje ne znamo na čemu su zasnovane. Predlog bi bio da za svaku tvrdnju uzetu iz knjige stavi referenca na poglavlje na koje se odnosi ili stranica u knjizi.

Još jedan predlog bi bio da u sažetku istaknu šta je motivacija, odnosno cilj ovog rada. \\
\odgovor{Smatramo da je jasno iz sadržaja sažetka da je cilj rada diskusija i opis nekih poznatih problema u oblasti primene veštačke inteligencije. Slažemo se da bi bilo lepše da je sažetak malo opširniji. Međutim, imali smo poteškoća kako uklopiti sažetak tako da naslov, sažetak i literatura stanu na prvu stranu, kako bismo ispoštovali formu.}


\section{Sitne primedbe}
% Напишете своја запажања на тему штампарских-стилских-језичких грешки
Reference su u tekstu postavljene na pogrešnom delu u rečenici. Treba da stoje unutar nje, pre tačke. \\
\odgovor{Izmenjeno je, sada su reference na pravom mestu, pre tačke.}

Ne stavlja se '','' ispred ''i'' i ''ili'' \\
\odgovor{Na par mesta je izbačeno ˝i˝ iza kojeg je zarez, a na par mesta je ostavljeno jer je u ulozi intenzifikatora, što je gramatički ispravno.}

2 Veštačke inteligencije i društvene mreže, slovna greška: ''preoporučivanja'' \\
\odgovor{Ispravljena je slovna greška.}

2.1.1 Youtube, 1. pasus, slovna greška: ''pretrežuje'' \\
\odgovor{Ispravljena je slovna greška.}


2.1.1 Youtube, 2. pasus, slovna greška: ''specijalana'' \\
\odgovor{Ispravljena je slovna greška.}


2.1.1 Youtube, 3. pasus, slovna greška: ''uspavanka'', a treba ''uspavanki'' \\
\odgovor{Ispravljena je slovna greška.}


2.1.2 Spotify, 2. pasus, umesto ''playlist-e'' napisati ''plejlista'' ili iskoristiti neki srpski prevod, ''lista pesama'' \\
\odgovor{Ispravljeno je po predlogu na lista pesama.}

2.2 Zavisnost, 1. pasus, možda umesto ''persona'' da se stavi ''osoba'' \\
\odgovor{Ispravljeno je po predlogu.}

2.2 Zavisnost, referenca [7] bi trebalo da stoji na kraju potpoglavlja, jer se osnosi na celo potpoglavlje. \\
\odgovor{Ispravljeno je.}

2.2 Zavisnost, Tabela 1 nije referisana iz teksta i u naslovu slovna greška ''izraženo'', a treba ''izražen''. \\
\odgovor{Dodata je referenca, i ispravljena je gramatička greška po predlogu.}

3. Autonomna vozila, 2.pasus, slovna greška: ''Poslednih'' \\
\odgovor{Ispravljena je slovna greška.}

3.1 Pojava etičkih problema, slovna greška: ''Poznavajući'', a treba ''Poznajući'' \\
\odgovor{Izmenjeno je jer je gramatički ispravno predloženo.}

3.2 Problem tramvaja, 2. pasus, slovna greška: ''ponšanja'' \\
\odgovor{Ispravljena je slovna greška.}

3.4 Prednosti autonomne vožnje, slovna greška: ''kontraverznih'', a treba ''kontroverznih'' \\
\odgovor{Ispravljena je slovna greška.}

4. Veštačka inteligencija u medicini, 1. pasus, slovna greška: ''primnene''

\odgovor{Ispravljena je slovna greška.}

4.1 Potrebni podaci, 1. pasus, slovna greška: ''algoritma'', a treba ''algoritama''

\odgovor{Ispravljena je slovna greška.}

4.1 Potrebni podaci, 3. pasus, slovna greška: ''korititi''

\odgovor{Ispravljena je slovna greška.}

4.1 Potrebni podaci, 5.pasus, slovna greška: ''ponašanem''

\odgovor{Ispravljena je slovna greška.}

4.2 Veštačka inteligencija i lekari, 4. pasus, slovna greška: ''činjecice''

\odgovor{Ispravljena je slovna greška.}

4.2 Veštačka inteligencija i lekari, 4. pasus, slovna greška: ''daje''

\odgovor{Ispravljena je slovna greška.}

5 Veštačka inteligencija u vojne svrhe, 2. pasus, slovna greška: ''ljutskih'', a treba ''ljudskih''

\odgovor{Ispravljena je slovna greška.}

5.1.1 Problem odgovornosti, 2. pasus, posle ''Naime'' ide zarez

\odgovor{Ispravljena je greška sa interpunkcijom.}

5.1.2 Problem pretprilagođenosti, 1.pasus, umesto ''krupni detalji'' je možda bolje napisati npr. ''bitni detalji''

\odgovor{Ispravljena je kontradikcija.}

5.2 Pozitivne strane vojne veštačke inteligencije, slovna greška: ''pretstavlja'', a treba ''predstavlja''

\odgovor{Ispravljena je slovna greška.}

5.2 Pozitivne strane vojne veštačke inteligencije, dva puta napisano ''su''

\odgovor{Obrisano je jedno ˝su˝.}

6 Zaklučak, 2.pasus, slovna greška: ''veštačke'', a treba ''veštačkoj''

\odgovor{Ispravljena je slovna greška.}

6 Zaključak, 2. pasus, drugu rečenicu malo preformulisati

\odgovor{Rečenica je preformulisana.}

\section{Provera sadržajnosti i forme seminarskog rada}
% Oдговорите на следећа питања --- уз сваки одговор дати и образложење

\begin{enumerate}
\item Da li rad dobro odgovara na zadatu temu?\\
Da, naslov se slaže sa onim što je u njemu pisano.

\item Da li je nešto važno propušteno?\\
Ne, pokrivene su sve teme i dodate još neke zanimljivosti.

\item Da li ima suštinskih grešaka i propusta?\\
Nema, rad zadovoljava sve zahteve, obrađene teme su pokrile temu rada i slažu se sa naslovom.

\item Da li je naslov rada dobro izabran?\\
Da, ono što je obrađeno u radu dobro odgovara datom naslovu.

\item Da li sažetak sadrži prave podatke o radu?\\
Sažetak sadrži glavne teme rade, ali bi mogla da se doda kratka motivacija, cilj rada.

\item Da li je rad lak-težak za čitanje?\\
Rad je lak za čitanje, poglavlja su lepo izdeljena i imaju tok.

\item Da li je za razumevanje teksta potrebno predznanje i u kolikoj meri?\\
Za ovaj rad nije potrebno nikakvo predznanje. Pojmovi koji su stručni su opisani na najnižem nivou tako da ih svi shvate.

\item Da li je u radu navedena odgovarajuća literatura?\\
Sva literatura je odgovarajuća i relevantna za datu temu. 

\item Da li su u radu reference korektno navedene?\\
Nisu, na mnogim mestima ne postoje reference i postavljali su na pogrešnom mestu u tekstu (treba pre tačke da stoje). \\
\odgovor{Izmenjeno je, sada su pravilno postavljene reference.}

\item Da li je struktura rada adekvatna?\\
Da, sadrži sve elemente seminarskog rada.

\item Da li rad sadrži sve elemente propisane uslovom seminarskog rada (slike, tabele, broj strana...)?\\
Da, sve uslovi su ispunjeni. Postoji jedna tabela, 2 slike, 12 strana, preko 7 referenci, i ima najmanje jedna referenca na knjigu, na naučni članak i na veb adresu.

\item Da li su slike i tabele funkcionalne i adekvatne?\\
Tabela 1. nije referisana u tekstu, ne zna se na koji deo se odnosi, a nije ni naznačeno odakle su te informacije. Slika 1. ne sadrži podatke za koje piše da sadrži (brojeve). \\
\odgovor{Izmenjen je opis slike kako bi odgovarao njenom sadržaju. Smatramo da nema potrebe dodavati brojeve jer nisu od preteranog značaja čitaocu, već odnos brojeva koji su vodeće države u ovoj oblasti dostigle, što je jasno na osnovu grafika. Takođe, dodata je referenca na tabelu.}

\end{enumerate}

\section{Ocenite sebe}
% Napišite koliko ste upućeni u oblast koju recenzirate: 
% a) ekspert u datoj oblasti
% b) veoma upućeni u oblast
% c) srednje upućeni
% d) malo upućeni 
% e) skoro neupućeni
% f) potpuno neupućeni
% Obrazložite svoju odluku
Srednje sam upućena u ovu oblast, jer je veštačka inteligencija oblast koju smo radili na fakultetu i upoznata sam sa njenom tehničkom stranom, ali što se tiče etičkih pitanja veštačke inteligencije, to je oblast kojom se nisam mnogo bavila niti istraživala. 

\chapter{Dodatak na recenziju}
Za navođenje naslova radova u uvodu, mislim da bi trebalo da naslov bude preveden na srpski, a u zagradi da se navede originalni naziv, ali nisam nigde pronašla odgovor šta je od te dve opcije ispravno, pa zato navodim ovde. Takođe mislim da bi engleske termine trebalo ispraviti, npr. umesto YouTube da piše
Jutjub (eng. YouTube), ali ni to nisam navela gore jer nisam sigurna šta je ispravno.

\odgovor{Naslovi radova i članaka nisu prevedeni na srpski jer su originalni naslovi na engleskom i nemaju zvaničan prevod na srpski. Napravljene su izmene po pitanju englicizama poput ˝Jutjub˝ i navedena su njihova prava engleska imena.}

Čini mi se da je na dosta mesta korišćen google translate jer su pojedine rečenice baš neprirodno napisane. Za one rečenice koje su nejasne ili nemaju smisla sam istakla pojedinačno da ih je potrebno ispraviti. Nisam sigurna da li je u redu dodati komentar da treba da prođu ponovo ceo tekst i preformulišu rečenice koje se mogu napisati na jednostavniji i prirodniji način, pošto je to možda samo moj subjektivni utisak.

\odgovor{Za izradu rada nije korišćen google translate. Izmenjene su rečenice koje su eksplicitno navedene u gore obrađenim recenzijama, ostatak teksta je proveren uz neke izmene.}

U potpoglavlju 2.1.1 Youtube, 4. pasus, 9. red, predložila bih da se izbaci ˝Srećom˝ jer rad treba da bude objktivan, sačinjen od činjenica i tvrdnji, bez subjektivnih osećanja. Nisam sigurna da li je ovo ispravna sugestija ili samo ja tako razmisljam, zbog čega sam ga navela ovde.

\odgovor{Slažemo se da rad treba da bude sačinjen od činjenica i tvrdnji, međutim u ovom kontekstu ne smatramo da je fokus stavljen na naša subjektivna osećanja sreće, već je reč upotrebljena više u vidu stilske slobode kako bi tekst bio interesantniji za čitanje.}

U potpoglavlju 5.1 Negativne strane vojne veštačke inteligencije, korišćen je pojam ˝pretprilagodenost˝ koji ne postoji. Potencijalno su tu hteli da iskoriste reč ˝preprilagodenost˝, ali nisam mogla da pronađem koji termin su pokušali da prevedu na srpski jezik.

\odgovor{Reč je trebala da bude ˝preprilagođenost˝ i adekvatno je izmenjena u tekstu.}

\chapter{Dodatne izmene}
%Ovde navedite ukoliko ima izmena koje ste uradili a koje vam recenzenti nisu tražili. 

\end{document}
