% !TEX encoding = UTF-8 Unicode
\documentclass[a4paper]{article}

\usepackage{color}
\usepackage{url}
\usepackage[T2A]{fontenc} % enable Cyrillic fonts
\usepackage[utf8]{inputenc} % make weird characters work
\usepackage{graphicx}

\usepackage[english,serbian]{babel}
%\usepackage[english,serbianc]{babel} %ukljuciti babel sa ovim opcijama, umesto gornjim, ukoliko se koristi cirilica

\usepackage[unicode]{hyperref}
\hypersetup{colorlinks,citecolor=green,filecolor=green,linkcolor=blue,urlcolor=blue}

\usepackage{listings}

%\newtheorem{primer}{Пример}[section] %ćirilični primer
\newtheorem{primer}{Primer}[section]

\definecolor{mygreen}{rgb}{0,0.6,0}
\definecolor{mygray}{rgb}{0.5,0.5,0.5}
\definecolor{mymauve}{rgb}{0.58,0,0.82}

\lstset{ 
  backgroundcolor=\color{white},   % choose the background color; you must add \usepackage{color} or \usepackage{xcolor}; should come as last argument
  basicstyle=\scriptsize\ttfamily,        % the size of the fonts that are used for the code
  breakatwhitespace=false,         % sets if automatic breaks should only happen at whitespace
  breaklines=true,                 % sets automatic line breaking
  captionpos=b,                    % sets the caption-position to bottom
  commentstyle=\color{mygreen},    % comment style
  deletekeywords={...},            % if you want to delete keywords from the given language
  escapeinside={\%*}{*)},          % if you want to add LaTeX within your code
  extendedchars=true,              % lets you use non-ASCII characters; for 8-bits encodings only, does not work with UTF-8
  firstnumber=1000,                % start line enumeration with line 1000
  frame=single,	                   % adds a frame around the code
  keepspaces=true,                 % keeps spaces in text, useful for keeping indentation of code (possibly needs columns=flexible)
  keywordstyle=\color{blue},       % keyword style
  language=Python,                 % the language of the code
  morekeywords={*,...},            % if you want to add more keywords to the set
  numbers=left,                    % where to put the line-numbers; possible values are (none, left, right)
  numbersep=5pt,                   % how far the line-numbers are from the code
  numberstyle=\tiny\color{mygray}, % the style that is used for the line-numbers
  rulecolor=\color{black},         % if not set, the frame-color may be changed on line-breaks within not-black text (e.g. comments (green here))
  showspaces=false,                % show spaces everywhere adding particular underscores; it overrides 'showstringspaces'
  showstringspaces=false,          % underline spaces within strings only
  showtabs=false,                  % show tabs within strings adding particular underscores
  stepnumber=2,                    % the step between two line-numbers. If it's 1, each line will be numbered
  stringstyle=\color{mymauve},     % string literal style
  tabsize=2,	                   % sets default tabsize to 2 spaces
  title=\lstname                   % show the filename of files included with \lstinputlisting; also try caption instead of title
}

\begin{document}
\section{Zaključak}
\label{sec: Zaključak}
Etički problemi u veštačkoj inteligenciji su brojni, opširni i veoma teški za rešavanje iz više razloga. Za početak oblasti od kojih su sačinjeni ovi problemi, etika i veštačka inteligencija, su same po sebi veoma kompleksne. Sa jedne strane, imamo etiku, nauku zasnovanu na pitanju morala koji se po svojoj definiciji razlikuje od pojedinca do pojedinca i nema precizno definisane ispravne i pogrešne odgovore na neka pitanja. Sa druge strane, imamo veštačku inteligenciju, relativno svestranu i kompleksnu granu programiranja koja obuhvata i pokušava da reši mnoštvo različitih, ponekad previše opštih, problema. Sa ovim na umu, postavlja se pitanje kako naučiti mašinu da etički rešava problem ako mi sami ne možemo da se usaglasimo po pitanju etike, a istovemeno ne možemo da sagledamo sve strane tog kompleksnog problema?
\newline
Problemi etike u veštačke inteligenciji su brojni i do sada spomenuti su samo neki od njih. Etički problemi su prisutni u praktično svakoj situaciji, ukoliko je sagledamo iz određenog ugla, neki od kojih su očigledniji od drugih. Pored obrađenih problema u oblastima društvenih mreža, autonomnih vozila, medicine i vojske postoje i brojne druge oblasti sa svojim problemima kao što su umetnost, radna snaga, obrada slike i zvuka, ekonomija i mnogi drugi za koje je potrebna detaljnija diskusija.
\end{document}